% LaTeX resume using res.cls
\documentclass[11pt]{res} % default is 10 pt
\usepackage{helvetica}
\usepackage{microtype}
%\setlength{\textheight}{10in} % increase text height to fit resume on 1 page
%\topmargin=-0.3in % start text higher on the page
\usepackage[colorlinks=true,urlcolor=blue]{hyperref}
\usepackage{comment}
\usepackage{geometry}
\geometry{a4paper, total={210mm,297mm}, left=20mm, right=40mm, top=20mm, bottom=20mm}
\newcommand{\highlight}[1]{\textbf{#1}}
\newcommand{\articletitle}[1]{\flushleft{\color{black} #1}}

\begin{document}

\name{Doctor Luke J. Shingles}

\begin{resume}

\section{Skills and interests}
  I am a scientist and software engineer focused on using software to solve complex problems on high-performance computers. Through my training in Software Engineering and experience in scientific research, I have been able to apply high-performance algorithms and data structures to build astrophysical simulations, while also developing skills in data analysis, visualisation, and technical communication through papers and presentations.

  Some my highlights have included writing high-performance parallel code to numerically solve systems of differential equations (e.g., to model chemical enrichment of stellar clusters and high-energy particle interactions), using matrix-based methods for solving large systems of linear equations (to determine atomic level populations), and applying Monte Carlo techniques to simulate photon interactions (radiative transfer) in three dimensions within material ejected from supernova explosions. The analysis of my theoretical simulations running on thousands of CPU cores and the comparison to real-world observations has led to research insights that have been published in peer-reviewed scientific journals.

  From my interest in parallel algorithms, I have naturally become excited about the use of GPUs to accelerate numerical problem-solving. My own experiments with NVIDIA CUDA have found that large performance benefits can be gained quickly, and even greater benefits can be achieved through further optimisation and a deeper understanding of the hardware. I am also interested in machine learning and computer vision, and look for opportunities to work with these technologies, to solve problems by developing optimised code for CPUs and GPUs, and to keep up with latest developments in parallel and heterogeneous computing.

  \href{https://www.linkedin.com/in/lukeshingles/}{LinkedIn Profile}

\section{Email}
  \href{mailto:l.shingles@qub.ac.uk}{l.shingles@qub.ac.uk}

\section{Citizenship}
  Australia, Ireland

\section{Languages}
  English (native), Portuguese (basic), Mandarin Chinese (basic)

\section{Address}
  Astrophysics Research Centre,\\
  School of Mathematics and Physics,\\
  Queen's University Belfast\\
  Belfast, Co. Antrim, BT7 1NN\\
  Northern Ireland, United Kingdom

\section{Programming experience}
  C, Fortran, Python (numpy/pandas/matplotlib), parallelism with CUDA, MPI, and OpenMP.\\
  Most of the scientific simulation codes I have worked on are not publicly available. However, some \href{https://github.com/lukeshingles/artistools}{plotting and analysis tools} I developed for the radiative transport code and the \href{https://github.com/lukeshingles/evelchemevol}{Fortran/OpenMP chemical evolution code} are public on my \href{https://github.com/lukeshingles}{GitHub Profile}.

\section{Education and recent experience}
  {\it Postdoctoral Research Fellow}, Queen's University Belfast, Aug 2015--present\\
  Working in the research group of Stuart Sim developing a 3D radiative transfer code to model Type Ia supernovae in their nebular phase (hundreds of days after explosion).

  {\it Doctor of Philosophy (Astrophysics)}, Australian National University, 2012--2015\\
  Thesis: \href{https://openresearch-repository.anu.edu.au/handle/1885/16507}{`Neutron-Capture Nucleosynthesis and the Chemical Evolution of Globular Clusters'}\\
  Department: Research School of Astronomy \& Astrophysics\\
  Primary Supervisor: Amanda Karakas\\
  Advisors: David Yong, Gary Da Costa, John Lattanzio (Monash), Richard Stancliffe (Bonn)

  {\it Bachelor of Science with Honours (First Class)}, Australian National University, 2008--2011\\
  Majors: Astronomy \& Astrophysics, Theoretical Physics, Mathematics\\
  Thesis: `The Sulfur Anomaly in Planetary Nebulae and Post-AGB Stars'\\
  Department: Research School of Astronomy \& Astrophysics\\
  Supervisor: Amanda Karakas
%  Honours grade: 86\%

%  {\it Bachelor of Science}, Australian National University, 2010\\
%  Grade average: 80\% (High Distinction)

  {\it Bachelor of Information Technology}, Queensland University of Technology, 2003--2007\\
  Major: Software Engineering

\section{Awards and Scholarships}
  RSAA Alex Rodgers Travelling Scholarship, 2014\\
  Astronomical Society of Australia Travel Assistance, 2014\\
  RSAA Honourable Mention for Best Student Paper Prize, 2013\\
  IAU Travel Grant for IAUS298, 2013\\
  Australian Postgraduate Award, 2012-2015\\
  International Year of Astronomy Honours Scholarship, 2011\\
  RSAA Summer Research Scholarship, 2010

\section{Service and Committees}
	QUB School of Maths and Physics Postdoctoral Society Representative, Jan 2016--present\\
	QUB School of Maths and Physics Athena SWAN committee for gender equality, Jan 2016--present\\
	QUB ARC Supernova Journal Club coordinator, Oct 2015--Oct 2016\\
	ANU RSAA Stellar Lunch coordinator, Feb 2014--Nov 2014\\
	ANU RSAA Computer Committee, Oct 2013--Apr 2015

\section{Technical Presentations}
  Contributed talk, The extragalactic explosive Universe: the new era of transient surveys and data-driven discovery, Garching, Germany, September 2019\\
  Contributed talk, Workshop on Radiative Transfer in Supernovae, Garching, Germany, August 2019\\
  Contributed talk, XIXth Workshop on Nuclear Astrophysics, Ringberg, Germany, March 2019\\
  Invited Colloquium, ASTRON Institute for Radio Astronomy, Dwingeloo, Netherlands, November 2018\\
  Contributed Talk, Radiation Transfer and Explosive Thermonuclear Burning in Supernovae, Rehovot, Israel, June 2018\\
  Poster Presentation, Supernovae From Simulations to Observations and Nucleosynthetic Fingerprints, Bad Honnef, Germany, January 2017\\
  Contributed Talk, Supernovae: The Outliers, Garching, Germany, September 2016\\
  Contributed Talk, RAS National Astronomy Meeting, Nottingham, UK, July 2016\\
  Contributed Talk, 18th Workshop on Nuclear Astrophysics, Ringberg, Germany, March 2016\\
  Group Talk at Stars Meeting, Institute of Astronomy, Cambridge, UK, Nov 2015\\
  Seminar, QUB, Belfast, UK, Oct 2015\\
  Contributed Talk, ASA AGM, Perth, Australia, July 2015\\
  Contributed Talk, ANITA Workshop, Canberra, Australia, Feb 2015\\
  Contributed Talk, Mount Stromlo Student Christmas Seminars, Canberra, Australia, Nov 2014\\
  Group Talk at Stars Meeting, Institute of Astronomy, Cambridge, UK, Sept 2014\\
  % Poster Presentation, Why Galaxies Care About AGB Stars, Vienna, Austria, July 2014\\
  Contributed Talk, Nucleosynthesis in AGB Stars, Bad Honnef, Germany, July 2014\\
  Contributed Talk, Overcoming Great Barriers in Galactic Archaeology II, Palm Cove, Australia, 2014\\
  Group Talk at Stellar Lunch, ANU RSAA, Australia, August 2013\\
  % Poster Presentation, IAUS298 Setting the Scene for GAIA and LAMOST, Lijiang, China, May 2013\\
  % Poster Presentation, Astronomical Society of Australia Meeting, Sydney, Australia, 2012\\
  % Poster Presentation, Astronomical Society of Australia Meeting, Adelaide, Australia, 2011

\section{Teaching Experience}
  {\it Level Four MSci Project} \hfill Queen's University Belfast\\
  \null\hfill Sept 2017 -- Jan 2018\\
  Co-supervised two MSci students with projects on positron emission from Type Ia supernovae
  and high-mass stellar evolution with helium-rich abundances.\\

  {\it PHY1001 Foundation Physics} \hfill Queen's University Belfast\\
  \null\hfill Oct 2017\\
  Presented lectures on circular motion and simple harmonic oscillators.\\

  {\it ANU-ASTRO2x Exoplanets} \hfill Australian National University\\
  \null\hfill Jun--Sep 2015\\
  Teaching assistant for edX online course run by Brian Schmidt and Paul Francis on exoplanet search techniques -- pulsar timing, radial-velocity variations, transits, microlensing, and direct imaging with adaptive optics.\\

  {\it ANU-ASTRO1x Greatest Unsolved Mysteries of the Universe}, \hfill Australian National University\\
  \null\hfill Mar--Jun 2015\\
  Teaching assistant for edX online course run by Brian Schmidt and Paul Francis covering the expanding universe, dark energy, dark matter, and gamma-ray bursts.\\

  {\it ASTR3007 From Stars to Galaxies} \hfill Australian National University\\
  \null\hfill Feb--Jun 2013 and May--Jun 2014\\
  Teaching assistant for the third-year course on stellar evolution \& nucleosynthesis, galactic structure \& dynamics, and introductory computer programming. Duties included marking assignments and answering student questions in the classroom.\\

  {\it PHYS1201 Physics 2} \hfill Australian National University\\
  \null\hfill Jul--Nov 2012 and Jul--Nov 2013\\
  Teaching assistant for first-year course covering introductory special relativity, electromagnetism, waves \& optics, and thermodynamics. Duties included marking assignments and answering student questions in the classroom.

\pagebreak
\begin{comment}
\section{Referees}
  \textbf{Dr. Stuart Sim}\\
  Lecturer, Centre for Astrophysics Research\\
  Queen's University Belfast, UK\\
  stuart.sim@anu.edu.au

  \textbf{Dr. Amanda Karakas}\\
  Lecturer, School of Physics & Astronomy\\
  Monash University, Australia\\
  amanda.karakas@monash.edu

  \textbf{Prof. Gary Da Costa}\\
  Professor, Research School of Astronomy \& Astrophysics\\
  Australian National University\\
  gary.dacosta@anu.edu.au
\end{comment}

\section{Refereed Journal Articles}

\articletitle{\href{}{Monte Carlo radiative transfer for the nebular phase of Type Ia supernovae}}\\
\highlight{L. Shingles}, S. A. Sim, M. Kromer, K. Maguire, M. Bulla, C. Collins, C. P. Ballance, A. S. Michel, C. A. Ramsbottom, F. K. R\"opke, I. R. Seitenzahl, N. B. Tyndall, 2019, MNRAS (submitted)

\articletitle{\href{}{A year-long plateau in the late-time near-infrared light curves of Type Ia supernovae}}\\
Or Graur, Kate Maguire, Russell Ryan, Matt Nicholl, Arturo Avelino, Adam G. Riess, \highlight{Luke Shingles}, Ivo R. Seitenzahl, and Robert Fisher, 2019, Nature Astronomy (accepted)

\articletitle{\href{http://adsabs.harvard.edu/abs/2018MNRAS.477.3567M}{Using late-time optical and near-infrared spectra to constrain Type Ia supernova explosion properties}}\\
K. Maguire, S. A. Sim, \highlight{L. Shingles}, J. Spyromilio, A. Jerkstrand, M. Sullivan, T.-W. Chen, R. Cartier, G. Dimitriadis, C. Frohmaier, L. Galbany, C. P. Gutiérrez, G. Hosseinzadeh, D. A. Howell, C. Inserra, R. Rudy, J. Sollerman, 2018, MNRAS

\articletitle{\href{http://adsabs.harvard.edu/abs/2017Natur.551...75S}{A kilonova as the electromagnetic counterpart to a gravitational-wave source}}\\
S. J. Smartt, T.-W. Chen, A.Jerkstrand, M. Coughlin, E. Kankare, S. A. Sim, M. Fraser, C. Inserra, K. Maguire, K. C. Chambers,
M. E. Huber, T. Kr\"uhler, G. Leloudas, M. Magee, \highlight{L. J. Shingles}, and 107 additional authors, 2017, Nature

\articletitle{\href{http://adsabs.harvard.edu/abs/2017ApJ...848L..12A}{Multi-messenger Observations of a Binary Neutron Star Merger}}\\
Joint-authored by several collaborations including ePESSTO (including \highlight{L. J. Shingles}), 2017, The Astrophysical Journal Letters

\articletitle{\href{http://adsabs.harvard.edu/abs/2017ApJ...837..176Y}{A chemical signature from fast-rotating low-metallicity massive stars: ROA 276 in omega Centauri}}\\
David Yong, John E. Norris, Gary S. Da Costa, Laura M. Stanford, Amanda I. Karakas, \highlight{Luke J. Shingles}, Raphael Hirschi, Marco Pignatari, 2017, ApJ, 837, 176

\articletitle{\href{http://adsabs.harvard.edu/abs/2015MNRAS.452.2804S}{Evolution and nucleosynthesis of helium-rich asymptotic giant branch models}}\\
\highlight{Luke J. Shingles}, Carolyn L. Doherty, Amanda I. Karakas, Richard J. Stancliffe, John C. Lattanzio, Maria Lugaro, 2015, MNRAS, 452, 2804

\articletitle{\href{http://adsabs.harvard.edu/abs/2015MNRAS.450..815M}{Iron and s-element abundance variations in NGC 5286: comparison with anomalous' globular clusters and Milky Way satellites}}\\
A. F. Marino, A. P. Milone, A. I. Karakas, L. Casagrande, D. Yong, \highlight{L. Shingles}, G. Da Costa, J. Norris, P. B. Stetson,  K. Lind, M. Asplund, R. Collet, H. Jerjen, L. Sbordone, A. Aparicio, \& S. Cassisi, 2015, MNRAS, 450, 815

\articletitle{\href{http://adsabs.harvard.edu/abs/2014ApJ...795...34S}{The s-process enrichment of the globular clusters M4 and M22}}\\
\highlight{Luke J. Shingles}, Amanda I. Karakas, Raphael Hirschi, Cherie K. Fishlock, David Yong, Gary S. Da Costa, \& Anna F. Marino, 2014, ApJ, 795, 34

\articletitle{\href{http://adsabs.harvard.edu/abs/2014MNRAS.441.3396Y}{Iron and neutron-capture element abundance variations in the globular cluster M2 (NGC 7089)}}\\
David Yong, Ian U. Roederer, Frank Grundahl, Gary S. Da Costa, Amanda I. Karakas, John E. Norris, Wako Aoki, Cherie K. Fishlock, A. F. Marino, A. P. Milone, \& \highlight{Luke J. Shingles}, 2014, MNRAS, 441, 3396

\articletitle{\href{http://adsabs.harvard.edu/abs/2013Ap\%26SS.347...47V}{Augmented reality in astrophysics}}\\
Fr\'{e}d\'{e}ric Vogt \& \highlight{Luke J. Shingles}, 2013, Ap\&SS, 347, 47

\articletitle{\href{http://adsabs.harvard.edu/abs/2013MNRAS.431.2861S}{Is the sulphur anomaly in planetary nebulae caused by the s-process?}}\\
\highlight{Luke J. Shingles} \& Amanda I. Karakas, 2013, MNRAS, 431, 2861
\end{resume}
\end{document}
