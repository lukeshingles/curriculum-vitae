%%%%%%%%%%%%%%%%%%%%%%%%%%%%%%%%%%%%%%%%%
% Wilson Resume/CV
% XeLaTeX Template
% Version 1.0 (22/1/2015)
%
% This template has been downloaded from:
% http://www.LaTeXTemplates.com
%
% Original author:
% Howard Wilson (https://github.com/watsonbox/cv_template_2004) with
% extensive modifications by Vel (vel@latextemplates.com)
%
% License:
% CC BY-NC-SA 3.0 (http://creativecommons.org/licenses/by-nc-sa/3.0/)
%
%%%%%%%%%%%%%%%%%%%%%%%%%%%%%%%%%%%%%%%%%

%----------------------------------------------------------------------------------------
%	PACKAGES AND OTHER DOCUMENT CONFIGURATIONS
%----------------------------------------------------------------------------------------

\documentclass[11pt]{article} % Default font size

%%%%%%%%%%%%%%%%%%%%%%%%%%%%%%%%%%%%%%%%%
% Wilson Resume/CV
% Structure Specification File
% Version 1.0 (22/1/2015)
%
% This file has been downloaded from:
% http://www.LaTeXTemplates.com
%
% License:
% CC BY-NC-SA 3.0 (http://creativecommons.org/licenses/by-nc-sa/3.0/)
%
%%%%%%%%%%%%%%%%%%%%%%%%%%%%%%%%%%%%%%%%%

%----------------------------------------------------------------------------------------
%	PACKAGES AND OTHER DOCUMENT CONFIGURATIONS
%----------------------------------------------------------------------------------------

\usepackage[a4paper, hmargin=25mm, vmargin=30mm, top=20mm]{geometry} % Use A4 paper and set margins

\usepackage{fancyhdr} % Customize the header and footer

\usepackage{lastpage} % Required for calculating the number of pages in the document

\usepackage[linkcolor=blue,colorlinks=true,urlcolor=blue]{hyperref} % Colors for links, text and headings

\setcounter{secnumdepth}{0} % Suppress section numbering

%\usepackage[proportional,scaled=1.064]{erewhon} % Use the Erewhon font
%\usepackage[erewhon,vvarbb,bigdelims]{newtxmath} % Use the Erewhon font
\usepackage[utf8]{inputenc} % Required for inputting international characters
\usepackage[T1]{fontenc} % Output font encoding for international characters

\usepackage{fontspec} % Required for specification of custom fonts
\setmainfont[Path = ./fonts/,
Extension = .otf,
BoldFont = Erewhon-Bold,
ItalicFont = Erewhon-Italic,
BoldItalicFont = Erewhon-BoldItalic,
SmallCapsFeatures = {Letters = SmallCaps}
]{Erewhon-Regular}

\usepackage{color} % Required for custom colors
\definecolor{slateblue}{rgb}{0.07,0.32,0.5}

\usepackage{sectsty} % Allows customization of titles
\sectionfont{\color{slateblue}} % Color section titles

\fancypagestyle{plain}{\fancyhf{}\cfoot{\thepage\ of \pageref{LastPage}}} % Define a custom page style
\pagestyle{plain} % Use the custom page style through the document
\renewcommand{\headrulewidth}{0pt} % Disable the default header rule
\renewcommand{\footrulewidth}{0pt} % Disable the default footer rule

\setlength\parindent{0pt} % Stop paragraph indentation

% Non-indenting itemize
\newenvironment{itemize-noindent}
{\setlength{\leftmargini}{0em}\begin{itemize}}
{\end{itemize}}

% Text width for tabbing environments
\newlength{\smallertextwidth}
\setlength{\smallertextwidth}{\textwidth}
\addtolength{\smallertextwidth}{-2cm}

\newcommand{\sqbullet}{~\vrule height 1ex width .8ex depth -.2ex} % Custom square bullet point definition

%----------------------------------------------------------------------------------------
%	MAIN HEADER COMMAND
%----------------------------------------------------------------------------------------

\renewcommand{\title}[1]{
{\huge{\color{slateblue}\textbf{#1}}}\\ % Header section name and color
\rule{\textwidth}{0.5mm}\\ % Rule under the header
}

%----------------------------------------------------------------------------------------
%	JOB COMMAND
%----------------------------------------------------------------------------------------

\newcommand{\job}[6]{
\begin{tabbing}
\hspace{2cm} \= \kill
\textbf{#1} \> \textbf{\href{#4}{#3}} \\
\textbf{#2} \>\+ \textit{#5} \\
\begin{minipage}{\smallertextwidth}
\vspace{2mm}
#6
\end{minipage}
\end{tabbing}
\vspace{2mm}
}

\newcommand{\article}[3]{
\item{\textbf{#1}\\
{#2}\\
\textit{#3}}
}

%----------------------------------------------------------------------------------------
%	INTERESTS GROUP COMMAND
%-----------------------------------------------------------------------------------------

\newcommand{\interestsgroup}[1]{
\begin{tabbing}
\hspace{5mm} \= \kill
#1
\end{tabbing}
\vspace{-10mm}
}

\newcommand{\interest}[1]{\sqbullet \> \textbf{#1}\\[3pt]} % Define a custom command for individual interests

%----------------------------------------------------------------------------------------
%	TABBED BLOCK COMMAND
%----------------------------------------------------------------------------------------

\newcommand{\tabbedblock}[1]{
\begin{tabbing}
\hspace{2cm} \= \hspace{10cm} \= \kill
#1
\end{tabbing}
} % Include the file specifying document layout

\newcommand{\highlight}[1]{\textbf{#1}}

%----------------------------------------------------------------------------------------

\begin{document}

%----------------------------------------------------------------------------------------
%	NAME AND CONTACT INFORMATION
%----------------------------------------------------------------------------------------

\title{Luke Shingles, PhD} % , BSc Hons, BInfoTech SoftEng QUT

%------------------------------------------------

\parbox{0.5\textwidth}{ % First block
\begin{tabbing} % Enables tabbing
\hspace{2.3cm} \= \hspace{4cm} \= \kill % Spacing within the block
{\bf Location} \> Belfast, Northern Ireland, UK\\
{\bf Citizenship} \> Australian, Irish\\
{\bf Email} \> \href{mailto:luke.shingles@gmail.com}{luke.shingles@gmail.com}
\end{tabbing}}
\hfill % Horizontal space between the two blocks
\parbox{0.5\textwidth}{ % Second block
\begin{tabbing} % Enables tabbing
\hspace{2.3cm} \= \hspace{4cm} \= \kill % Spacing within the block
{\bf LinkedIn} \> \href{https://www.linkedin.com/in/lukeshingles}{linkedin.com/in/lukeshingles}\\
{\bf GitHub} \> \href{https://github.com/lukeshingles}{github.com/lukeshingles}\\
% {\bf Mobile} \> +44 7462 625 608
\end{tabbing}}

%----------------------------------------------------------------------------------------
%	PERSONAL PROFILE
%----------------------------------------------------------------------------------------
\section{Personal Statement}
I am an experienced researcher in astrophysics, with skills in scientific software development, data analysis, and visualisation. Some of the research software engineering highlights have included writing high-performance parallel code to numerically solve systems of differential equations (to model chemical enrichment of stellar clusters and high-energy particle interactions), using matrix-based methods for solving large systems of linear equations (to determine atomic level populations), and applying Monte Carlo techniques to simulate photon interactions (radiative transfer) in three dimensions. I have contributed to open source software used in research, and advocated for software engineering best-practices in collaborative projects.\\

I have a strong track record of obtaining scientific insights from computationally demanding theoretical work, having given over many scientific talks at professional conferences and institutes, and published articles in peer-reviewed science journals.\\

I have contributed as a co-investigator on successful research grants and HPC computing time proposals for international facilities. I enjoy teaching and have chosen to gain experience tutoring and giving lectures for undergraduate physics courses, and co-supervised PhD, Master's, and Google Summer of Code students on a variety of projects in astrophysics and software development.\\
%
% Within the School of Mathematics and Physics, I have volunteered as a Postdoctoral Representatives and am a member of the Athena SWAN committee for gender equality. In these roles, I have organised social events to build connections between people in different research clusters, organised career information days with invited speakers, and supported efforts to improve diversity in STEM careers.\\


I have expertise in:
\begin{itemize}
  \item Scientific research and publishing in peer-reviewed journals
  \item Leading the development of numerically-intensive C++ code for radiative transfer simulations
  \item Performance optimisation with multithreading and parallel algorithms (OpenMP and MPI)
  \item Collaborative development with version control (Git), continuous integration, and automated testing
  \item Developing tools to process and visualise large data sets to extract meaningful insights (Python with polars, pandas, and matplotlib)
  \item Statistics, differential equations, and linear algebra (machine learning fundamentals)
\end{itemize}



\section{Education and Employment}

\job
{Oct 2021--}{}
{Postdoctoral Researcher}
{}
{GSI Helmholzzentrum für Schwerionenforschung, Darmstadt, Germany}
{\begin{itemize-noindent}
\item{Improved the ARTIS radiative transfer code with relativistic corrections and increased numerical accuracy to model kilonovae in three dimensions. Published in scientific journals, including Astrophysical Journal Letters.}
\end{itemize-noindent}
\textbf{Technologies:} C++, Python (Numpy/Polars/Pandas/Matplotlib), Rust, Git, OpenMP, MPI\\
}

\job
{Aug 2015--}{Sept 2021}
% {Astrophysicist and Software Engineer}
{Postdoctoral Researcher}
{}
{Queen's University Belfast, Northern Ireland}
{\begin{itemize-noindent}
\item{Lead developer on a radiative transfer code for large-scale simulations ($\sim$50k core-hour) and associated set of analysis/visualisation tools used by a group of researchers.}
\item{Implemented several matrix-based numerical solvers and Monte Carlo statistical estimators to model the physical conditions and radiation transport in supernovae.}
\item{Presented conference talks and lectures, wrote research papers, reports, and grant applications, supervised Masters and PhD students, and volunteered as the Postdoctoral Representative for the School of Mathematics \& Physics.}
\item{Built the \href{https://fallingstar-data.com/forcedphot/}{ATLAS Forced Photometry service}, which has over 1150 registered users.}
\end{itemize-noindent}
\textbf{Technologies:} C, Python (Numpy/Pandas/Matplotlib), Git, OpenMP, MPI\\
}

\job{2012--2015}{}
{Doctor of Philosophy (Astrophysics)}
{}{Australian National University}
{
\href{https://openresearch-repository.anu.edu.au/handle/1885/16507}{Thesis: Neutron-Capture Nucleosynthesis and the Chemical Evolution of Globular Clusters}\\
Primary Supervisor: Dr. Amanda Karakas
% Advisors: David Yong, Gary Da Costa, John Lattanzio (Monash), Richard Stancliffe (Bonn)\\
\begin{itemize-noindent}
  \item Computed numerical simulations of low-mass stars on Linux-based high-performance compute clusters (NCI Raijin system).
  \item Developed Fortran/OpenMP code to solve a system of differential equations to model chemical production in galaxies.
  \item Published insights from simulation results in peer-reviewed journals with implications for the evolution of low-mass stars and the origins of chemical elements in the universe.
  \item Teaching assistant for courses on first-year physics, third-year astrophysics, and online courses on cosmology and exoplanets.
\end{itemize-noindent}
\textbf{Technologies:} Fortran, OpenMP, Git, Python (Matplotlib), Mathematica
}

\job{2008--2011}{}
{Bachelor of Science with Honours (First Class)}
{}{Australian National University}
{
Honours Thesis: The Sulfur Anomaly in Planetary Nebulae and Post-AGB Stars\\
% Department: Research School of Astronomy \& Astrophysics\\
% Honours Supervisor: Amanda Karakas\\
Honours grade: 86\% (First Class)\\
Majors: Astronomy \& Astrophysics, Theoretical Physics, Mathematics\\
Course grade average: 80\% (High Distinction)\\
Selected results:
\begin{tabbing}\hspace{5pt} \= \hspace{9cm} \= \kill
\>\+\textit{Games, Graphs, and Machines} \> 85\% High Distinction\\
\textit{Maths Methods 1 Honours: Ordinary differential}\\\textit{\quad equations and advanced vector calculus} \> 85\% High Distinction \\
\textit{Maths Methods 2 Honours: Partial differential}\\\textit{\quad equations, Fourier analysis, and complex analysis} \> 78\% Distinction \\
\textit{Theoretical Physics} \> 87\% High Distinction \\
\textit{Applied Algebra 1 Honours: Groups}\\\textit{\quad rings, and advanced linear algebra} \> 78\% Distinction \\
\textit{Number theory and cryptography} \> 83\% High Distinction
\end{tabbing}
}

\job{2003-2007}{}
{Bachelor of Information Technology}
{}{Queensland University of Technology}
{
Major: Software Engineering
}

% \section{Awards and Scholarships}
%   RSAA Alex Rodgers Travelling Scholarship, 2014\\
%   Astronomical Society of Australia Travel Assistance, 2014\\
%   RSAA Honourable Mention for Best Student Paper Prize, 2013\\
%   IAU Travel Grant for Lijiang, China, IAUS298, 2013\\
%   Australian Postgraduate Award PhD Scholarship, 2012-2015\\
%   International Year of Astronomy Honours Scholarship, 2011\\
%   RSAA Summer Research Scholarship, 2010

% \section{Service and Committees}
%   QUB School of Maths and Physics Postdoctoral Society Representative, Jan 2016--2020\\
%   QUB School of Maths and Physics Athena SWAN committee for gender equality, Jan 2016--2020\\
%   QUB School of Maths and Physics Careers event committee, 2018, 2019\\
%   QUB ARC Supernova Journal Club coordinator, Oct 2015--Oct 2016\\
%   ANU RSAA Stellar Lunch coordinator, Feb 2014--Nov 2014\\
%   ANU RSAA Computer Committee, Oct 2013--Apr 2015
%
% \section{Technical Presentations}
%   Invited seminar, AEI Potsdam, Germany, September 2023\\
%   Contributed talk, ELEMENTS, Germany, May 2022\\
%   Contributed talk, FAIR next generation scientists - 7th Edition Workshop, Greece, May 2022\\
%   Plenary talk, Supernova Workshop at Heidelberg Institute of Theoretical Studies, Heidelberg, Germany, December 2019\\
%   Contributed talk, The extragalactic explosive Universe: the new era of transient surveys and data-driven discovery, Garching, Germany, September 2019\\
%   Contributed talk, Workshop on Radiative Transfer in Supernovae, Garching, Germany, August 2019\\
%   Contributed talk, XIXth Workshop on Nuclear Astrophysics, Ringberg, Germany, March 2019\\
%   Invited Colloquium, ASTRON Institute for Radio Astronomy, Dwingeloo, Netherlands, November 2018\\
%   Contributed Talk, Radiation Transfer and Explosive Thermonuclear Burning in Supernovae, Rehovot, Israel, June 2018\\
%   % Poster Presentation, Supernovae From Simulations to Observations and Nucleosynthetic Fingerprints, Bad Honnef, Germany, January 2017\\
%   Contributed Talk, Supernovae: The Outliers, Garching, Germany, September 2016\\
%   Contributed Talk, RAS National Astronomy Meeting, Nottingham, UK, July 2016\\
%   Contributed Talk, 18th Workshop on Nuclear Astrophysics, Ringberg, Germany, March 2016\\
%   Group Talk at Stars Meeting, Institute of Astronomy, Cambridge, UK, Nov 2015\\
%   Seminar, QUB, Belfast, UK, Oct 2015\\
%   Contributed Talk, ASA AGM, Perth, Australia, July 2015\\
%   Contributed Talk, ANITA Workshop, Canberra, Australia, Feb 2015\\
%   Contributed Talk, Mount Stromlo Student Christmas Seminars, Canberra, Australia, Nov 2014\\
%   Group Talk at Stars Meeting, Institute of Astronomy, Cambridge, UK, Sept 2014\\
%   % Poster Presentation, Why Galaxies Care About AGB Stars, Vienna, Austria, July 2014\\
%   Contributed Talk, Nucleosynthesis in AGB Stars, Bad Honnef, Germany, July 2014\\
%   Contributed Talk, Overcoming Great Barriers in Galactic Archaeology II, Palm Cove, Australia, 2014\\
%   Group Talk at Stellar Lunch, ANU RSAA, Australia, August 2013\\
%   % Poster Presentation, IAUS298 Setting the Scene for GAIA and LAMOST, Lijiang, China, May 2013\\
%   % Poster Presentation, Astronomical Society of Australia Meeting, Sydney, Australia, 2012\\
%   % Poster Presentation, Astronomical Society of Australia Meeting, Adelaide, Australia, 2011

% \section{Teaching Experience}
%   {\it Level Four MSci Project} \hfill Queen's University Belfast\\
%   \null\hfill Sept 2017 -- Jan 2018\\
%   Co-supervised two MSci students with projects on positron emission from Type Ia supernovae
%   and high-mass stellar evolution with helium-rich abundances.\\
%
%   {\it PHY1001 Foundation Physics} \hfill Queen's University Belfast\\
%   \null\hfill Oct 2017\\
%   Presented lectures on circular motion and simple harmonic oscillators.\\
%
%   {\it ANU-ASTRO2x Exoplanets} \hfill Australian National University\\
%   \null\hfill Jun--Sep 2015\\
%   Teaching assistant for edX online course run by Brian Schmidt and Paul Francis on exoplanet search techniques -- pulsar timing, radial-velocity variations, transits, microlensing, and direct imaging with adaptive optics.\\
%
%   {\it ANU-ASTRO1x Greatest Unsolved Mysteries of the Universe}, \hfill Australian National University\\
%   \null\hfill Mar--Jun 2015\\
%   Teaching assistant for edX online course run by Brian Schmidt and Paul Francis covering the expanding universe, dark energy, dark matter, and gamma-ray bursts.\\
%
%   {\it ASTR3007 From Stars to Galaxies} \hfill Australian National University\\
%   \null\hfill Feb--Jun 2013 and May--Jun 2014\\
%   Teaching assistant for the third-year course on stellar evolution \& nucleosynthesis, galactic structure \& dynamics, and introductory computer programming. Duties included marking assignments and answering student questions in the classroom.\\
%
%   {\it PHYS1201 Physics 2} \hfill Australian National University\\
%   \null\hfill Jul--Nov 2012 and Jul--Nov 2013\\
%   Teaching assistant for first-year course covering introductory special relativity, electromagnetism, waves \& optics, and thermodynamics. Duties included marking assignments and answering student questions in the classroom.

% \section{References}
%   \textbf{Dr. Stuart Sim}\\
%   Reader, Centre for Astrophysics Research\\
%   Queen's University Belfast, UK\\
%   s.sim@qub.ac.uk
%

\section{Refereed Journal Articles}
\begin{itemize-noindent}

\article{Nebular \lbrack Fe II\rbrack\ emission as a constraint on Type Ia supernova progenitors}
{\highlight{L. Shingles}, Stuart Sim, Andreas Floers, et al.}{Monthly Notices of the Royal Astronomical Society, (2021, in preparation).}

\article{\href{https://ui.adsabs.harvard.edu/abs/2020MNRAS.499.4312K/abstract}{The influence of line opacity treatment in STELLA on supernova light curves}}
{A. Kozyreva, \highlight{L. Shingles}, Alexey Mironov, Petr Baklanov, Sergey Blinnikov}
{Monthly Notices of the Royal Astronomical Society, Volume 499, Issue 3, pp.4312-4324 (2020).}

\article{\href{https://ui.adsabs.harvard.edu/abs/2020MNRAS.492.2029S}{Monte Carlo radiative transfer for the nebular phase of Type Ia supernovae}}
{\highlight{L. Shingles}, S. A. Sim, M. Kromer, K. Maguire, M. Bulla, C. Collins, C. P. Ballance, A. S. Michel, C. A. Ramsbottom, F. K. R\"opke, I. R. Seitenzahl, N. B. Tyndall}
{Monthly Notices of the Royal Astronomical Society, Volume 492, Issue 2, p.2029-2043 (2020).}

\article{\href{https://ui.adsabs.harvard.edu/abs/2019NatAs.tmp..463G}{A year-long plateau in the late-time near-infrared light curves of Type Ia supernovae}}
{Or Graur, Kate Maguire, Russell Ryan, Matt Nicholl, Arturo Avelino, Adam G. Riess, \highlight{Luke Shingles}, Ivo R. Seitenzahl, and Robert Fisher}
{Nature Astronomy, Advanced Online Publication (2019).}

\article{\href{http://adsabs.harvard.edu/abs/2018MNRAS.477.3567M}{Using late-time optical and near-infrared spectra to constrain Type Ia supernova explosion properties}}
{K. Maguire, S. A. Sim, \highlight{L. Shingles}, J. Spyromilio, A. Jerkstrand, M. Sullivan, T.-W. Chen, R. Cartier, G. Dimitriadis, C. Frohmaier, L. Galbany, C. P. Gutiérrez, G. Hosseinzadeh, D. A. Howell, C. Inserra, R. Rudy, J. Sollerman}
{Monthly Notices of the Royal Astronomical Society, Volume 477, Issue 3, p.3567-3582 (2018).}

\article{\href{http://adsabs.harvard.edu/abs/2017Natur.551...75S}{A kilonova as the electromagnetic counterpart to a gravitational-wave source}}
{S. J. Smartt, T.-W. Chen, A.Jerkstrand, M. Coughlin, E. Kankare, S. A. Sim, M. Fraser, C. Inserra, K. Maguire, K. C. Chambers,
M. E. Huber, T. Kr\"uhler, G. Leloudas, M. Magee, \highlight{L. J. Shingles}, and 107 additional authors}
{Nature, Volume 551, Issue 7678, pp. 75-79 (2017)}

\article{\href{http://adsabs.harvard.edu/abs/2017ApJ...848L..12A}{Multi-messenger Observations of a Binary Neutron Star Merger}}
{Joint-authored by several collaborations including ePESSTO (including \highlight{L. J. Shingles})}
{The Astrophysical Journal Letters, Volume 848, Issue 2, article id. L12, 59 pp. (2017).}

\article{\href{http://adsabs.harvard.edu/abs/2017ApJ...837..176Y}{A chemical signature from fast-rotating low-metallicity massive stars: ROA 276 in omega Centauri}}
{David Yong, John E. Norris, Gary S. Da Costa, Laura M. Stanford, Amanda I. Karakas, \highlight{Luke J. Shingles}, Raphael Hirschi, Marco Pignatari}
{The Astrophysical Journal, Volume 837, Issue 2, article id. 176, 8 pp. (2017).}

\article{\href{http://adsabs.harvard.edu/abs/2015MNRAS.452.2804S}{Evolution and nucleosynthesis of helium-rich asymptotic giant branch models}}
{\highlight{Luke J. Shingles}, Carolyn L. Doherty, Amanda I. Karakas, Richard J. Stancliffe, John C. Lattanzio, Maria Lugaro}
{Monthly Notices of the Royal Astronomical Society, Volume 452, Issue 3, p.2804-2821 (2015).}

\article{\href{http://adsabs.harvard.edu/abs/2015MNRAS.450..815M}{Iron and s-element abundance variations in NGC 5286: comparison with anomalous' globular clusters and Milky Way satellites}}
{A. F. Marino, A. P. Milone, A. I. Karakas, L. Casagrande, D. Yong, \highlight{L. Shingles}, G. Da Costa, J. Norris, P. B. Stetson,  K. Lind, M. Asplund, R. Collet, H. Jerjen, L. Sbordone, A. Aparicio, \& S. Cassisi}
{Monthly Notices of the Royal Astronomical Society, Volume 450, Issue 1, p.815-845 (2015).}

\article{\href{http://adsabs.harvard.edu/abs/2014ApJ...795...34S}{The s-process enrichment of the globular clusters M4 and M22}}
{\highlight{Luke J. Shingles}, Amanda I. Karakas, Raphael Hirschi, Cherie K. Fishlock, David Yong, Gary S. Da Costa, \& Anna F. Marino}
{The Astrophysical Journal, Volume 795, Issue 1, article id. 34, 12 pp. (2014).}

\article{\href{http://adsabs.harvard.edu/abs/2014MNRAS.441.3396Y}{Iron and neutron-capture element abundance variations in the globular cluster M2 (NGC 7089)}}
{David Yong, Ian U. Roederer, Frank Grundahl, Gary S. Da Costa, Amanda I. Karakas, John E. Norris, Wako Aoki, Cherie K. Fishlock, A. F. Marino, A. P. Milone, \& \highlight{Luke J. Shingles}}
{Monthly Notices of the Royal Astronomical Society, Volume 441, Issue 4, p.3396-3416 (2014).}

\article{\href{http://adsabs.harvard.edu/abs/2013Ap\%26SS.347...47V}{Augmented reality in astrophysics}}
{Fr\'{e}d\'{e}ric Vogt \& \highlight{Luke J. Shingles}}
{Astrophysics and Space Science, Volume 347, Issue 1, pp.47-60 (2013).}

\article{\href{http://adsabs.harvard.edu/abs/2013MNRAS.431.2861S}{Is the sulphur anomaly in planetary nebulae caused by the s-process?}}
{\highlight{Luke J. Shingles} \& Amanda I. Karakas}
{Monthly Notices of the Royal Astronomical Society, Volume 431, Issue 3, p.2861-2871 (2013).}



\end{itemize-noindent}

%----------------------------------------------------------------------------------------
%	REFEREE SECTION
%----------------------------------------------------------------------------------------

% \section{Referees}
%
% \parbox{0.5\textwidth}{ % First block
% \begin{tabbing}
% \hspace{2.75cm} \= \hspace{4cm} \= \kill % Spacing within the block
% {\bf Name} \> Bill Lumbergh \\ % Referee name
% {\bf Company} \> Initech Inc. \\ % Referee company
% {\bf Position} \> Vice President \\ % Referee job title
% {\bf Contact} \> \href{mailto:bill@initech.com}{bill@initech.com} % Referee contact information
% \end{tabbing}}
% \hfill % Horizontal space between the two blocks
% \parbox{0.5\textwidth}{ % Second block
% \begin{tabbing}
% \hspace{2.75cm} \= \hspace{4cm} \= \kill % Spacing within the block
% {\bf Name} \> Michael "Big Mike" Tucker\\ % Referee name
% {\bf Company} \> Burbank Buy More \\ % Referee company
% {\bf Position} \> Store Manager \\ % Referee job title
% {\bf Contact} \> \href{mailto:mike@buymore.com}{mike@buymore.com} % Referee contact information
% \end{tabbing}}

%----------------------------------------------------------------------------------------

\end{document}
