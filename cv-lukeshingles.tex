% LaTeX resume using res.cls
\documentclass[11pt]{res} % default is 10 pt
\usepackage{helvetica}
\usepackage{microtype}
%\setlength{\textheight}{10in} % increase text height to fit resume on 1 page
%\topmargin=-0.3in % start text higher on the page
\usepackage[colorlinks=true,urlcolor=blue]{hyperref}
\usepackage{comment}
\usepackage{geometry}
\geometry{a4paper, total={210mm,297mm}, left=20mm, right=40mm, top=20mm, bottom=20mm}

\begin{document}

\name{Doctor Luke J. Shingles}

\begin{resume}

\section{Skills and interests}
   Scientific research using high-performance computing clusters, Monte Carlo simulations, radiative transfer,
   supernovae, nucleosynthesis, stellar evolution, chemical evolution

\section{Email}
  \href{mailto:l.shingles@qub.ac.uk}{l.shingles@qub.ac.uk}

\section{Citizenship}
  Australian, Irish

\section{Languages}
  English (native), Portuguese (basic), Mandarin Chinese (basic)

\section{Address}
  Astrophysics Research Centre,\\
  School of Mathematics and Physics,\\
  Queen's University Belfast\\
  Belfast, Co. Antrim, BT7 1NN\\
  Northern Ireland, United Kingdom

\section{Programming language/API experience}
  C, Fortran, Python (numpy/pandas/matplotlib), MPI, OpenMP\\
  \href{https://github.com/lukeshingles}{https://github.com/lukeshingles}

\section{Education and employment}
  {\it Postdoctoral Research Fellow}, Queen's University Belfast, 2015--present\\
  Working in the research group of Stuart Sim developing a 3D radiative transfer code for application to Type Ia supernovae hundreds of days after explosion.

  {\it Doctor of Philosophy (Astrophysics)}, Australian National University, 2012--2015\\
  Thesis: \href{https://openresearch-repository.anu.edu.au/handle/1885/16507}{`Neutron-Capture Nucleosynthesis and the Chemical Evolution of Globular Clusters'}\\
  Department: Research School of Astronomy \& Astrophysics\\
  Primary Supervisor: Amanda Karakas\\
  Advisors: David Yong, Gary Da Costa, John Lattanzio (Monash), Richard Stancliffe (Bonn)

  {\it Bachelor of Science with Honours (First Class)}, Australian National University, 2008--2011\\
  Majors: Astronomy \& Astrophysics, Theoretical Physics, Mathematics\\
  Thesis: `The Sulfur Anomaly in Planetary Nebulae and Post-AGB Stars'\\
  Department: Research School of Astronomy \& Astrophysics\\
  Supervisor: Amanda Karakas
%  Honours grade: 86\%

%  {\it Bachelor of Science}, Australian National University, 2010\\
%  Grade average: 80\% (High Distinction)

  {\it Bachelor of Information Technology}, Queensland University of Technology, 2003--2007\\
  Major: Software Engineering

\section{Awards and Scholarships}
  RSAA Alex Rodgers Travelling Scholarship, 2014\\
  Astronomical Society of Australia Travel Assistance, 2014\\
  RSAA Honourable Mention for Best Student Paper Prize, 2013\\
  IAU Travel Grant for IAUS298, 2013\\
  Australian Postgraduate Award, 2012-2015\\
  International Year of Astronomy Honours Scholarship, 2011\\
  RSAA Summer Research Scholarship, 2010

\section{Service and Committees}
	QUB School of Maths and Physics Postdoctoral Society Representative, Jan 2016--present\\
	QUB ARC Supernova Journal Club coordinator, Oct 2015--Oct 2016\\
	ANU RSAA Stellar Lunch coordinator, Feb 2014--Nov 2014\\
	ANU RSAA Computer Committee, Oct 2013--Apr 2015

\section{Talks and Poster Presentations}
  Contributed Talk, Radiation Transfer and Explosive Thermonuclear Burning in Supernovae, Rehovot, Israel, June 2018\\
  Poster Presentation, Supernovae - From Simulations to Observations and Nucleosynthetic Fingerprints, Bad Honnef, Germany, January 2017\\
  Contributed Talk, Supernovae: The Outliers, Garching, Germany, September 2016\\
  Contributed Talk, RAS National Astronomy Meeting, Nottingham, UK, July 2016\\
  Contributed Talk, 18th Workshop on Nuclear Astrophysics, Ringberg, Germany, March 2016\\
  Group Talk at Stars Meeting, Institute of Astronomy, Cambridge, UK, Nov 2015\\
  Seminar, QUB, Belfast, UK, Oct 2015\\
  Contributed Talk, ASA AGM, Perth, Australia, July 2015\\
  Contributed Talk, ANITA Workshop, Canberra, Australia, Feb 2015\\
  Contributed Talk, Mount Stromlo Student Christmas Seminars, Canberra, Australia, Nov 2014\\
  Group Talk at Stars Meeting, Institute of Astronomy, Cambridge, UK, Sept 2014\\
  Poster Presentation, Why Galaxies Care About AGB Stars, Vienna, Austria, July 2014\\
  Contributed Talk, Nucleosynthesis in AGB Stars, Bad Honnef, Germany, July 2014\\
  Contributed Talk, Overcoming Great Barriers in Galactic Archaeology II, Palm Cove, Australia, 2014\\
  Group Talk at Stellar Lunch, ANU RSAA, Australia, August 2013\\
  Poster Presentation, IAUS298 Setting the Scene for GAIA and LAMOST, Lijiang, China, May 2013\\
  Poster Presentation, Astronomical Society of Australia Meeting, Sydney, Australia, 2012\\
  Poster Presentation, Astronomical Society of Australia Meeting, Adelaide, Australia, 2011

\section{Teaching Experience}
  {\it Level Four MSci Project} \hfill Queen's University Belfast\\
  \null\hfill Sept 2017 -- Jan 2018\\
  Co-supervised two students with projects on positron emission from Type Ia supernovae,
  and high-mass stellar evolution with helium-rich abundances.\\

  {\it PHY1001 Foundation Physics} \hfill Queen's University Belfast\\
  \null\hfill Oct 2017\\
  Presented two lectures on circular motion and simple harmonic oscillators.\\

  {\it ANU-ASTRO2x Exoplanets} \hfill Australian National University\\
  \null\hfill Jun--Sep 2015\\
  Teaching assistant for edX online course run by Brian Schmidt and Paul Francis on exoplanet search techniques -- pulsar timing, radial-velocity variations, transits, microlensing, and direct imaging with adaptive optics.\\

  {\it ANU-ASTRO1x Greatest Unsolved Mysteries of the Universe}, \hfill Australian National University\\
  \null\hfill Mar--Jun 2015\\
  Teaching assistant for edX online course run by Brian Schmidt and Paul Francis covering the expanding universe, dark energy, dark matter, and gamma-ray bursts.\\

  {\it ASTR3007 From Stars to Galaxies} \hfill Australian National University\\
  \null\hfill Feb--Jun 2013 and May--Jun 2014\\
  Teaching assistant for the third-year course on stellar evolution \& nucleosynthesis, galactic structure \& dynamics, and introductory computer programming. Duties included marking assignments and answering student questions in the classroom.\\

  {\it PHYS1201 Physics 2} \hfill Australian National University\\
  \null\hfill Jul--Nov 2012 and Jul--Nov 2013\\
  Teaching assistant for first-year course covering introductory special relativity, electromagnetism, waves \& optics, and thermodynamics. Duties included marking assignments and answering student questions in the classroom.

%\pagebreak
\begin{comment}
\section{Referees}
  \textbf{Dr. Stuart Sim}\\
  Lecturer, Centre for Astrophysics Research\\
  Queen's University Belfast, UK\\
  stuart.sim@anu.edu.au

  \textbf{Dr. Amanda Karakas}\\
  Lecturer, School of Physics & Astronomy\\
  Monash University, Australia\\
  amanda.karakas@monash.edu

  \textbf{Prof. Gary Da Costa}\\
  Professor, Research School of Astronomy \& Astrophysics\\
  Australian National University\\
  gary.dacosta@anu.edu.au

\end{comment}
\end{resume}
\end{document}
